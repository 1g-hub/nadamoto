%\documentstyle[epsf,twocolumn]{jarticle}       %LaTeX2.09仕様
\documentclass[onecolumn]{jarticle}     %pLaTeX2e仕様
%%%%%%%%%%%%%%%%%%%%%%%%%%%%%%%%%%%%%%%%%%%%%%%%%%%%%%%%%%%%%%
%%
%%  基本 バージョン
%%
%%%%%%%%%%%%%%%%%%%%%%%%%%%%%%%%%%%%%%%%%%%%%%%%%%%%%%%%%%%%%%%%
\setlength{\topmargin}{-45pt}
%\setlength{\oddsidemargin}{0cm} 
\setlength{\oddsidemargin}{-7.5mm}
%\setlength{\evensidemargin}{0cm} 
\setlength{\textheight}{24.1cm}
%setlength{\textheight}{25cm} 
\setlength{\textwidth}{17.4cm}
%\setlength{\textwidth}{172mm} 
\setlength{\columnsep}{11mm}

\kanjiskip=.07zw plus.5pt minus.5pt


%【節がかわるごとに(1.1)(1.2) …(2.1)(2.2)と数式番号をつけるとき】
%\makeatletter
%\renewcommand{\theequation}{%
%\thesection.\arabic{equation}} %\@addtoreset{equation}{section}
%\makeatother

%\renewcommand{\arraystretch}{0.95} 行間の設定
\newcommand{\hanwa}{IO$\exists$K}
%%%%%%%%%%%%%%%%%%%%%%%%%%%%%%%%%%%%%%%%%%%%%%%%%%%%%%%%
\usepackage[dvipdfmx]{graphicx}   %pLaTeX2e仕様(要\documentstyle ->\documentclass)
\usepackage{url}
\usepackage{hhline}
\usepackage{multirow}
\usepackage{textcomp}
\usepackage{mathcomp}
\makeatletter
\newcommand{\figcaption}[1]{\def\@captype{figure}\caption{#1}}
\newcommand{\tblcaption}[1]{\def\@captype{table}\caption{#1}}
\makeatother
%%%%%%%%%%%%%%%%%%%%%%%%%%%%%%%%%%%%%%%%%%%%%%%%%%%%%%%%

\begin{document}

%bibtex用の設定
\bibliographystyle{jplain}

\noindent

\hspace{1em}
令和 3 年 1 月 8 日(金)
ゼミ資料
\hfill
B4 灘本 紗也佳

\vspace{2mm}

\hrule

\begin{center}
{\Large \bf 進捗報告}
\end{center}


\hrule
\vspace{3mm}

% ここから 文章 Start!
\section{今週やったこと}
	\begin{itemize}
	  \item 画像変換のネットワークの作成
       \vspace{-0.5em}
 	  \item 卒論の執筆
	\end{itemize}

\section{今週の収穫}
\subsection{画像変換のネットワークの作成}
\begin{figure}[h]
    \centering
    \begin{tabular}{c}
        \begin{minipage}{0.9\hsize}
            \centering
            \includegraphics[clip, width=16cm]{img.jpg}
	       \caption{ネットワーク概要図}
	       \label{data_generator}
        \end{minipage}\\

\vspace{1zh}				

        \begin{minipage}{0.9\hsize}
            \centering
            \includegraphics[clip, width=16cm]{img2.jpg}
	        \caption{ネットワーク概要図(変更版)}
	        \label{data_generator2}
        \end{minipage}
    \end{tabular}
\end{figure}

昨年作ろうとしていた画像変換ネットワーク (図 \ref{data_generator}) だが, 写真風のなにかは生成されそうであるが, 入力の情報が失われてしまう危険性が高い. 
ネットで見つけたネットワーク (図 \ref{data_generator2}) のほうが良さそうだったのでそちらでやっていくことにした. 
リモート先のパソコンのネットワークを誤って切ってしまったので, 別の資料に生成結果を載せる. この生成された画像は, 写真と星図の 2 クラス分類を学習させた VGG16 で分類を行ったところ, 写真であると分類された. 

今回, オリオン座の星図 287 枚を学習用に, 残り 1 枚をテスト用画像にした. 最終的に完成した変換ネットワークに, 学習に使用した星図を通してもいいのか...? また各星座につきそれぞれ一つずつ変換ネットワークを作るべきなのか...? 

\subsection{卒論の執筆}
ぼちぼち卒論の執筆も進めていきます. 

\section{今後の方針}
変換ネットワークをいじりつつ, 卒論も進めていきます. 


% \bibliography{2020_05_08_nadamoto} 
% \bibliographystyle{unsrt} 

\end{document}